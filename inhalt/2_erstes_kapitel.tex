\section{Grundlagen}

\newglossaryentry{latex}{
    name=LaTeX,
    description={Ein Textsatzsystem für wissenschaftliche Arbeiten}
}
\newacronym{dhbw}{DHBW}{Duale Hochschule Baden-Württemberg}

Hier ein Beispiel mit \gls{latex} und der Abkürzung \gls{dhbw}.\vglcite{webseite2025}

Dieses Kapitel erläutert grundlegenden Begriffe, Technologien und Rahmenbedingungen, die für das Verständnis der Arbeit erforderlich sind.

\subsection{Begriffserklärungen}
Zur besseren Einordnung des Themas werden in diesem Abschnitt zentrale Begriffe definiert, die im weiteren Verlauf der Arbeit verwendet werden.

\subsection{Relevante Technologien}
Hier werden die Technologien, Methoden oder Werkzeuge beschrieben, die in der Arbeit Anwendung finden, beispielsweise:

\begin{itemize}
    \item Webtechnologien wie HTML, CSS und JavaScript
    \item Frameworks wie Laravel oder Vue.js
    \item Datenbanksysteme (z.\,B. MySQL)
\end{itemize}

\subsection{Bisheriger Forschungsstand}
Der Stand der Technik und relevante wissenschaftliche Arbeiten werden vorgestellt und kritisch betrachtet.\vglcite[15\psq]{muster2025}